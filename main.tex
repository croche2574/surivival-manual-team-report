\documentclass{article}
    % General document formatting
    \usepackage[margin=1in]{geometry}
    \usepackage[parfill]{parskip}
    \usepackage[utf8]{inputenc}
    \usepackage{textcomp}
    \usepackage{gensymb}
    \usepackage{titling}
    
    \usepackage[backend=bibtex8, style=ieee]{biblatex}
    \addbibresource{citations.bib}
    
    \title{Potential Temperature Outputs of Solar Concentration Methods}
    \author{The Outlierz}
    \date{\today}

\begin{document}

\maketitle

\pagenumbering{arabic}

\section{Introduction}
    \paragraph{}
    The purpose of this report is to examine the temperature ranges that different solar concentration methods can 
    produce and assess the feasibility of heat production on a residential scale. 
    Solar concentration is the process of using things such as mirrors, reflective material, or 
    lenses to focus large amounts of sunlight on to a smaller area.  This allows for varying amounts 
    of heat to be produced, ranging from around 90 \degree C to well over 4000 \degree C. 
    The heat generated using a solar concentrator can be used in many applications including power, air 
    conditioning, cooking, and boiling water.
    The primary applications that this report focuses on relate to heating water and other materials.
    This report analyzes three studies of solar concentration applications and the data those studies present.

\section{Method}
    \paragraph{}
    Solar thermal energy is a relatively new renewable source of heat.  On an industrial scale, one of the simplest forms of solar heat 
    production is called a linear trough.  Large parabolic mirrors are arranged in a north/south orientation and focus sunlight along a 
    pipe containing water or another fluid.  This liquid is then used to generate steam to run a turbine or for other industrial purposes.
    The Nevada Solar One plant is a large linear trough array capable of generating up to 134 million Kilowatt-hours per year.  With a cost 
    of \$266 million dollars, that works out to around \$4.15 per watt.  This is similar to the cost of a regular photovoltaic installation,
    but with the added benefit of thermal storage, allowing for energy production after the sun sets.  This suggests that solar thermal energy
    production is at least as cost effective and efficient on a large scale as solar photovoltaic.\cite{Basking-in-the-Sun}

    \paragraph{}
    Compound parabolic concentrating solar collectors (CPCs) are a very efficient form of solar thermal production in terms of space
    and heat production. In a direct comparison between CPCs and flat-plate collector (FPC) systems for radiant floor heating, water was
    heated to a maximum temperature of 95 \degree C, 25 \degree C higher than the FPC system.  This was accomplished with a much 
    smaller physical footprint, making CPCs a very useful form of solar thermal concentration.\cite{CPCs-vs-FPCs}
    
    \paragraph{}
    Another useful solar thermal system is Parabolic dish solar energy concentrators.  These systems can be constucted fairly easily,
    with a low cost overhead.
    The experiment using these concentrators dealt with phase change material (PCM) based 
    heat packs. The packs were placed in a black container filled with water at the focal point of the concentrator and allowed to absorb heat
    until they reached an average temperature of 39 \degree C. This process took approximately 35 minutes. 
    The packs then released heat over the time of 50 minutes. 
    This experiment showed that solar collectors can be used to generate and store heat for comfort and therapeutic applications.\cite{doi:10.1063/1.4949125}
    
\section{Results}
    \paragraph{}
    Across the three papers referenced, solar collectors are shown to be an efficient source of heat for various 
    industrial and small scale applications.  The linear trough design is one of the easiest to produce on a large scale 
    because the mirrors are easier to produce and the array only needs to be moved along one axis.  The average efficiency 
    of a linear trough system is between \%15 and \%30, with an average temperature of up to 400 degrees Celsius\cite{Basking-in-the-Sun}.  The parabolic 
    dish design was capable of producing temperatures of up to 505 degrees Celsius, and boiled water in less than an hour.  
    The overall footprint of this design was much smaller at 1.5 meters squared\cite{doi:10.1063/1.4949125}.  The final design we looked at was the CPC, 
    used in this case to provide underfloor heating.  This collector design was able to efficiently raise the temperature of 
    water faster and to higher temperatures than flat panel designs.  These results demonstrate the feasibility of solar 
    concentration based heat generation on a residential scale.

    \printbibliography
\end{document}